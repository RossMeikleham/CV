%%%%%%%%%%%%%%%%%%%%%%%%%%%%%%%%%%%%%%%%%
% Wilson Resume/CV
% XeLaTeX Template
% Version 1.0 (22/1/2015)
%
% This template has been downloaded from:
% http://www.LaTeXTemplates.com
%
% Original author:
% Howard Wilson (https://github.com/watsonbox/cv_template_2004) with
% extensive modifications by Vel (vel@latextemplates.com)
%
% License:
% CC BY-NC-SA 3.0 (http://creativecommons.org/licenses/by-nc-sa/3.0/)
%
%%%%%%%%%%%%%%%%%%%%%%%%%%%%%%%%%%%%%%%%%

%----------------------------------------------------------------------------------------
%	PACKAGES AND OTHER DOCUMENT CONFIGURATIONS
%----------------------------------------------------------------------------------------



\documentclass[10pt]{article} % Default font size

%%%%%%%%%%%%%%%%%%%%%%%%%%%%%%%%%%%%%%%%%
% Wilson Resume/CV
% Structure Specification File
% Version 1.0 (22/1/2015)
%
% This file has been downloaded from:
% http://www.LaTeXTemplates.com
%
% License:
% CC BY-NC-SA 3.0 (http://creativecommons.org/licenses/by-nc-sa/3.0/)
%
%%%%%%%%%%%%%%%%%%%%%%%%%%%%%%%%%%%%%%%%%

%----------------------------------------------------------------------------------------
%	PACKAGES AND OTHER DOCUMENT CONFIGURATIONS
%----------------------------------------------------------------------------------------

\usepackage[a4paper, hmargin=25mm, vmargin=30mm, top=20mm]{geometry} % Use A4 paper and set margins

\usepackage{fancyhdr} % Customize the header and footer

\usepackage{lastpage} % Required for calculating the number of pages in the document

\usepackage{hyperref} % Colors for links, text and headings

\setcounter{secnumdepth}{0} % Suppress section numbering

%\usepackage[proportional,scaled=1.064]{erewhon} % Use the Erewhon font
%\usepackage[erewhon,vvarbb,bigdelims]{newtxmath} % Use the Erewhon font
\usepackage[utf8]{inputenc} % Required for inputting international characters
\usepackage[T1]{fontenc} % Output font encoding for international characters

\usepackage{fontspec} % Required for specification of custom fonts
\setmainfont[Path = ./fonts/,
Extension = .otf,
BoldFont = Erewhon-Bold,
ItalicFont = Erewhon-Italic,
BoldItalicFont = Erewhon-BoldItalic,
SmallCapsFeatures = {Letters = SmallCaps}
]{Erewhon-Regular}

\usepackage{color} % Required for custom colors
\definecolor{slateblue}{rgb}{0.17,0.22,0.34}

\usepackage{sectsty} % Allows customization of titles
\sectionfont{\color{slateblue}} % Color section titles

\fancypagestyle{plain}{\fancyhf{}\cfoot{\thepage\ of \pageref{LastPage}}} % Define a custom page style
\pagestyle{plain} % Use the custom page style through the document
\renewcommand{\headrulewidth}{0pt} % Disable the default header rule
\renewcommand{\footrulewidth}{0pt} % Disable the default footer rule

\setlength\parindent{0pt} % Stop paragraph indentation

% Non-indenting itemize
\newenvironment{itemize-noindent}
{\setlength{\leftmargini}{0em}\begin{itemize}}
{\end{itemize}}

% Text width for tabbing environments
\newlength{\smallertextwidth}
\setlength{\smallertextwidth}{\textwidth}
\addtolength{\smallertextwidth}{-2cm}

\newcommand{\sqbullet}{~\vrule height 1ex width .8ex depth -.2ex} % Custom square bullet point definition

%----------------------------------------------------------------------------------------
%	MAIN HEADER COMMAND
%----------------------------------------------------------------------------------------

\renewcommand{\title}[1]{
{\huge{\color{slateblue}\textbf{#1}}}\\ % Header section name and color
\rule{\textwidth}{0.5mm}\\ % Rule under the header
}

%----------------------------------------------------------------------------------------
%	JOB COMMAND
%----------------------------------------------------------------------------------------

\newcommand{\job}[6]{
\begin{tabbing}
\hspace{2cm} \= \kill
\textbf{#1} \> \href{#4}{#3} \\
\textbf{#2} \>\+ \textit{#5} \\
\begin{minipage}{\smallertextwidth}
\vspace{2mm}
#6
\end{minipage}
\end{tabbing}
\vspace{2mm}
}


\newcommand{\jobtwo}[7]{
\begin{tabbing}
\hspace{2cm} \= \kill
\textbf{#1} \> \href{#4}{#3} \\
\textbf{#2} \>\+ \textit{#5} \\
\textit{#6} \\   %\>\+ \textit{#6} \\
\begin{minipage}{\smallertextwidth}
\vspace{2mm}
#7
\end{minipage}
\end{tabbing}
\vspace{2mm}
}

%----------------------------------------------------------------------------------------
%	SKILL GROUP COMMAND
%----------------------------------------------------------------------------------------

\newcommand{\skillgroup}[2]{
\begin{tabbing}
\hspace{5mm} \= \kill
\sqbullet \>\+ \textbf{#1} \\
\begin{minipage}{\smallertextwidth}
\vspace{2mm}
#2
\end{minipage}
\end{tabbing}
}

%----------------------------------------------------------------------------------------
%	INTERESTS GROUP COMMAND
%-----------------------------------------------------------------------------------------

\newcommand{\interestsgroup}[1]{
\begin{tabbing}
\hspace{5mm} \= \kill
#1
\end{tabbing}
\vspace{-10mm}
}

\newcommand{\interest}[1]{\sqbullet \> \textbf{#1}\\[3pt]} % Define a custom command for individual interests

%----------------------------------------------------------------------------------------
%	TABBED BLOCK COMMAND
%----------------------------------------------------------------------------------------

\newcommand{\tabbedblock}[1]{
\begin{tabbing}
\hspace{2cm} \= \hspace{4cm} \= \kill
#1
\end{tabbing}
}
 % Include the file specifying document layout

\usepackage{enumitem}

%----------------------------------------------------------------------------------------

\begin{document}

%----------------------------------------------------------------------------------------
%	NAME AND CONTACT INFORMATION
%----------------------------------------------------------------------------------------

\title{Ross Meikleham -- CV} % Print the main header

%------------------------------------------------

\parbox{0.5\textwidth}{ % First block
\begin{tabbing} % Enables tabbing
\hspace{3cm} \= \hspace{4cm} \= \kill % Spacing within the block
{\bf Address} \> 2/2 293 St Georges Road,\\ % Address line 1
\> Glasgow, G3 6JQ \\ % Address line 2
{\bf Date of Birth} \> 8$^{th}$ November 1992 \\ % Date of birth 

\end{tabbing}}
\hfill % Horizontal space between the two blocks
\parbox{0.5\textwidth}{ % Second block
\begin{tabbing} % Enables tabbing
\hspace{3cm} \= \hspace{4cm} \= \kill % Spacing within the block
{\bf Mobile Phone} \> (+44) (0) 7587220100 \\ % Mobile phone
{\bf Email} \> \href{mailto:rossmeikleham@hotmail.co.uk}{RossMeikleham@hotmail.co.uk} \\ % Email address
{\bf Github} \> \href{https://github.com/RossMeikleham}{https://github.com/RossMeikleham} \\ %Github profile
{\bf LinkedIn} \> \href{www.linkedin.com/in/RossMeikleham}{www.linkedin.com/in/RossMeikleham} % LinkedIn
\end{tabbing}}
%----------------------------------------------------------------------------------------
%	PERSONAL PROFILE
%----------------------------------------------------------------------------------------
\section{Education}

\job
{Sep 2011 -}{Present}
{University Of Glasgow}
{http://www.gla.ac.uk/}
{Master in Science (MSci) Computing Science and Mathematics}
{Expected to graduate July 2016\\\\
Awards:
\begin{itemize}
\item{O’Reilly Academic Prize for Best Overall Performance in Assessed Coursework in level 1 Computing Science in Session 2011-12}
\item{Level 4 Combined Honours Class Prize}
\end{itemize}}

\job
{Sep 2009 -}{Jul 2011}
{Shenley Brook End Sixth Form}
{http://www.sbeschool.org.uk/}
{A Levels}
{Grades:
\begin{itemize}[noitemsep]
\item{\textit{Computing} - A*}
\item{\textit{Mathematics} - A*}
\item{\textit{Further Mathematics} - A}
\end{itemize}
Awards:
\begin{itemize}[noitemsep]
\item{Young Programmer Of The Year 2010 - 2011}
\end{itemize}
}

%----------------------------------------------------------------------------------------
%	EMPLOYMENT HISTORY SECTION
%----------------------------------------------------------------------------------------

\section{Work Experience}

\job
{Oct 2015 -}{Dec 2015}
{University Of Glasgow}
{http://www.gla.ac.uk/}
{Assignment Marker}
{I aided the school of Mathematics and Statistics in marking and providing feedback in exercises for second year students in 
 their pure mathematics and linear algebra courses.}


\job
{Jun 2015 -}{Aug 2015}
{Toshiba Medical Visualization Systems Europe (TMVSE)}
{http://www.tmvse.com/}
{Software Engineering Intern}
{Over the course of my 13 week internship at Toshiba I developed a performance monitoring tool
written in C++ for one of their key infrastructure systems. This was intended to help pinpoint specific areas of their system which could be causing potental bottlenecks.
\\
\rule{0mm}{5mm}\textbf{Technologies:} C++, Javascript, Qt, Mercurial, Visual Studio.}

%------------------------------------------------

\job
{Nov 2013 -}{Apr 2014}
{Farmgeek}
{https://farmer.io/}
{Android Developer Intern}
{During my 3rd year at Glasgow University I worked 1 day a week at Farmgeek helping to develop a prototype Android application called Agricountant. This application was a stock control management system for livestock farms,that helps farmers measure their farms' performance and improve animal welfare standards.
\\
\rule{0mm}{5mm}\textbf{Technologies:} Java, Android, Git, Gradle, Linux.}


\job
{Jul 2012 -}{Aug 2012}
{Gulliver's Theme Parks}
{https://www.gulliversfun.co.uk/milton-keynes}
{Ride Operator}
{Over the summer of 2012 I worked as a Ride Operator at Gulliver's Land, this involved a number of responsibilities, including: 
\begin{itemize}[noitemsep]
\item{Following device maintenance and safety procedures to ensure the safety of guests on the rides.}
\item{Assisting guests with any queries they may have.}
\item{Responding to emergency situations effectively and reporting these to a manager in a timely manner.}
\end{itemize}
}

%----------------------------------------------------------------------------------------
%	PROJECTS SECTION
%----------------------------------------------------------------------------------------


\section{Open Source Projects}
\job
{2014 - }{Current}
{Gameboy Color Emulator}
{https://github.com/RossMeikleham/PlutoBoy}
{https://github.com/RossMeikleham/PlutoBoy}
{A multi-platform Gameboy Color emulator written in C with SDL; built as a hobby project. 
It currently runs on Linux, Windows, OSX, Android, iOS, and Sony PSP platforms. 
It's also able to run on web browsers using the Emscripten LLVM to Javascript compiler.} 


\jobtwo
{2014 - }{2015}
{Design And Compilation Of A Front End Language For The Glasgow Parallel Reduction Machine}
{}
{https://github.com/RossMeikleham/GPC (Code Repository)}
{https://github.com/RossMeikleham/Dissertation (Project Dissertation Repository)}
{For my honours year project at the University of Glasgow I designed a language called GPC; which is a subset of C++11 extended with a couple of new keywords to specify parallel or sequential evaluation of statements in blocks of code. The language is essentially a parallel evaluated communication language which manages composition of ordinary C++ code. I wrote the compiler using Haskell.}



%----------------------------------------------------------------------------------------
%    SKILLS SECTION
%----------------------------------------------------------------------------------------
\section{Skills}
\bf{Languages}
\begin{itemize}
\item \textit{Proficient In} - C, C++, Java, Haskell
\item \textit{Familiar With}  - Python, Javascript, Matlab, Idris, Pascal
\end{itemize}

\bf{Software Engineering}
\begin{itemize}
\item \textit{Version Control} - Git, Mercurial
\item \textit{Unit Testing} - JUnit(Java), HUnit(Haskell), Qt Test(C++)
\item \textit{Continuous Integration} - Travis CI, Appveyor (Both used for personal/open source projects), Jenkins (used in software development university course)
\item \textit{Agile Methodologies} - Test Driven Development(TDD)
\end{itemize}

\bf{Linux}
\begin{itemize}
\item \textit{Distributions} - Ubuntu, Fedora, Arch Linux 
\item \textit{Shells} - Bash, Zsh
\end{itemize}

\bf{Other}
\begin{itemize}
\item \textit{Frameworks} - Qt, SDL
\item \textit{Other Technologies} - Vim, \LaTeX
\end{itemize}

%----------------------------------------------------------------------------------------
%	INTERESTS SECTION
%----------------------------------------------------------------------------------------

\section{Interests}
Functional Programming, Open Source Development, Linux, Parallel Programming\\\\


%----------------------------------------------------------------------------------------
%	REFEREE SECTION
%----------------------------------------------------------------------------------------


\textit{References can be provided upon request}

%----------------------------------------------------------------------------------------

\end{document}
